
% Default to the notebook output style

    


% Inherit from the specified cell style.




    
\documentclass[11pt]{article}

    
    
    \usepackage[T1]{fontenc}
    % Nicer default font (+ math font) than Computer Modern for most use cases
    \usepackage{mathpazo}

    % Basic figure setup, for now with no caption control since it's done
    % automatically by Pandoc (which extracts ![](path) syntax from Markdown).
    \usepackage{graphicx}
    % We will generate all images so they have a width \maxwidth. This means
    % that they will get their normal width if they fit onto the page, but
    % are scaled down if they would overflow the margins.
    \makeatletter
    \def\maxwidth{\ifdim\Gin@nat@width>\linewidth\linewidth
    \else\Gin@nat@width\fi}
    \makeatother
    \let\Oldincludegraphics\includegraphics
    % Set max figure width to be 80% of text width, for now hardcoded.
    \renewcommand{\includegraphics}[1]{\Oldincludegraphics[width=.8\maxwidth]{#1}}
    % Ensure that by default, figures have no caption (until we provide a
    % proper Figure object with a Caption API and a way to capture that
    % in the conversion process - todo).
    \usepackage{caption}
    \DeclareCaptionLabelFormat{nolabel}{}
    \captionsetup{labelformat=nolabel}

    \usepackage{adjustbox} % Used to constrain images to a maximum size 
    \usepackage{xcolor} % Allow colors to be defined
    \usepackage{enumerate} % Needed for markdown enumerations to work
    \usepackage{geometry} % Used to adjust the document margins
    \usepackage{amsmath} % Equations
    \usepackage{amssymb} % Equations
    \usepackage{textcomp} % defines textquotesingle
    % Hack from http://tex.stackexchange.com/a/47451/13684:
    \AtBeginDocument{%
        \def\PYZsq{\textquotesingle}% Upright quotes in Pygmentized code
    }
    \usepackage{upquote} % Upright quotes for verbatim code
    \usepackage{eurosym} % defines \euro
    \usepackage[mathletters]{ucs} % Extended unicode (utf-8) support
    \usepackage[utf8x]{inputenc} % Allow utf-8 characters in the tex document
    \usepackage{fancyvrb} % verbatim replacement that allows latex
    \usepackage{grffile} % extends the file name processing of package graphics 
                         % to support a larger range 
    % The hyperref package gives us a pdf with properly built
    % internal navigation ('pdf bookmarks' for the table of contents,
    % internal cross-reference links, web links for URLs, etc.)
    \usepackage{hyperref}
    \usepackage{longtable} % longtable support required by pandoc >1.10
    \usepackage{booktabs}  % table support for pandoc > 1.12.2
    \usepackage[inline]{enumitem} % IRkernel/repr support (it uses the enumerate* environment)
    \usepackage[normalem]{ulem} % ulem is needed to support strikethroughs (\sout)
                                % normalem makes italics be italics, not underlines
    

    
    
    % Colors for the hyperref package
    \definecolor{urlcolor}{rgb}{0,.145,.698}
    \definecolor{linkcolor}{rgb}{.71,0.21,0.01}
    \definecolor{citecolor}{rgb}{.12,.54,.11}

    % ANSI colors
    \definecolor{ansi-black}{HTML}{3E424D}
    \definecolor{ansi-black-intense}{HTML}{282C36}
    \definecolor{ansi-red}{HTML}{E75C58}
    \definecolor{ansi-red-intense}{HTML}{B22B31}
    \definecolor{ansi-green}{HTML}{00A250}
    \definecolor{ansi-green-intense}{HTML}{007427}
    \definecolor{ansi-yellow}{HTML}{DDB62B}
    \definecolor{ansi-yellow-intense}{HTML}{B27D12}
    \definecolor{ansi-blue}{HTML}{208FFB}
    \definecolor{ansi-blue-intense}{HTML}{0065CA}
    \definecolor{ansi-magenta}{HTML}{D160C4}
    \definecolor{ansi-magenta-intense}{HTML}{A03196}
    \definecolor{ansi-cyan}{HTML}{60C6C8}
    \definecolor{ansi-cyan-intense}{HTML}{258F8F}
    \definecolor{ansi-white}{HTML}{C5C1B4}
    \definecolor{ansi-white-intense}{HTML}{A1A6B2}

    % commands and environments needed by pandoc snippets
    % extracted from the output of `pandoc -s`
    \providecommand{\tightlist}{%
      \setlength{\itemsep}{0pt}\setlength{\parskip}{0pt}}
    \DefineVerbatimEnvironment{Highlighting}{Verbatim}{commandchars=\\\{\}}
    % Add ',fontsize=\small' for more characters per line
    \newenvironment{Shaded}{}{}
    \newcommand{\KeywordTok}[1]{\textcolor[rgb]{0.00,0.44,0.13}{\textbf{{#1}}}}
    \newcommand{\DataTypeTok}[1]{\textcolor[rgb]{0.56,0.13,0.00}{{#1}}}
    \newcommand{\DecValTok}[1]{\textcolor[rgb]{0.25,0.63,0.44}{{#1}}}
    \newcommand{\BaseNTok}[1]{\textcolor[rgb]{0.25,0.63,0.44}{{#1}}}
    \newcommand{\FloatTok}[1]{\textcolor[rgb]{0.25,0.63,0.44}{{#1}}}
    \newcommand{\CharTok}[1]{\textcolor[rgb]{0.25,0.44,0.63}{{#1}}}
    \newcommand{\StringTok}[1]{\textcolor[rgb]{0.25,0.44,0.63}{{#1}}}
    \newcommand{\CommentTok}[1]{\textcolor[rgb]{0.38,0.63,0.69}{\textit{{#1}}}}
    \newcommand{\OtherTok}[1]{\textcolor[rgb]{0.00,0.44,0.13}{{#1}}}
    \newcommand{\AlertTok}[1]{\textcolor[rgb]{1.00,0.00,0.00}{\textbf{{#1}}}}
    \newcommand{\FunctionTok}[1]{\textcolor[rgb]{0.02,0.16,0.49}{{#1}}}
    \newcommand{\RegionMarkerTok}[1]{{#1}}
    \newcommand{\ErrorTok}[1]{\textcolor[rgb]{1.00,0.00,0.00}{\textbf{{#1}}}}
    \newcommand{\NormalTok}[1]{{#1}}
    
    % Additional commands for more recent versions of Pandoc
    \newcommand{\ConstantTok}[1]{\textcolor[rgb]{0.53,0.00,0.00}{{#1}}}
    \newcommand{\SpecialCharTok}[1]{\textcolor[rgb]{0.25,0.44,0.63}{{#1}}}
    \newcommand{\VerbatimStringTok}[1]{\textcolor[rgb]{0.25,0.44,0.63}{{#1}}}
    \newcommand{\SpecialStringTok}[1]{\textcolor[rgb]{0.73,0.40,0.53}{{#1}}}
    \newcommand{\ImportTok}[1]{{#1}}
    \newcommand{\DocumentationTok}[1]{\textcolor[rgb]{0.73,0.13,0.13}{\textit{{#1}}}}
    \newcommand{\AnnotationTok}[1]{\textcolor[rgb]{0.38,0.63,0.69}{\textbf{\textit{{#1}}}}}
    \newcommand{\CommentVarTok}[1]{\textcolor[rgb]{0.38,0.63,0.69}{\textbf{\textit{{#1}}}}}
    \newcommand{\VariableTok}[1]{\textcolor[rgb]{0.10,0.09,0.49}{{#1}}}
    \newcommand{\ControlFlowTok}[1]{\textcolor[rgb]{0.00,0.44,0.13}{\textbf{{#1}}}}
    \newcommand{\OperatorTok}[1]{\textcolor[rgb]{0.40,0.40,0.40}{{#1}}}
    \newcommand{\BuiltInTok}[1]{{#1}}
    \newcommand{\ExtensionTok}[1]{{#1}}
    \newcommand{\PreprocessorTok}[1]{\textcolor[rgb]{0.74,0.48,0.00}{{#1}}}
    \newcommand{\AttributeTok}[1]{\textcolor[rgb]{0.49,0.56,0.16}{{#1}}}
    \newcommand{\InformationTok}[1]{\textcolor[rgb]{0.38,0.63,0.69}{\textbf{\textit{{#1}}}}}
    \newcommand{\WarningTok}[1]{\textcolor[rgb]{0.38,0.63,0.69}{\textbf{\textit{{#1}}}}}
    
    
    % Define a nice break command that doesn't care if a line doesn't already
    % exist.
    \def\br{\hspace*{\fill} \\* }
    % Math Jax compatability definitions
    \def\gt{>}
    \def\lt{<}
    % Document parameters
    \title{ads}
    
    
    

    % Pygments definitions
    
\makeatletter
\def\PY@reset{\let\PY@it=\relax \let\PY@bf=\relax%
    \let\PY@ul=\relax \let\PY@tc=\relax%
    \let\PY@bc=\relax \let\PY@ff=\relax}
\def\PY@tok#1{\csname PY@tok@#1\endcsname}
\def\PY@toks#1+{\ifx\relax#1\empty\else%
    \PY@tok{#1}\expandafter\PY@toks\fi}
\def\PY@do#1{\PY@bc{\PY@tc{\PY@ul{%
    \PY@it{\PY@bf{\PY@ff{#1}}}}}}}
\def\PY#1#2{\PY@reset\PY@toks#1+\relax+\PY@do{#2}}

\expandafter\def\csname PY@tok@w\endcsname{\def\PY@tc##1{\textcolor[rgb]{0.73,0.73,0.73}{##1}}}
\expandafter\def\csname PY@tok@c\endcsname{\let\PY@it=\textit\def\PY@tc##1{\textcolor[rgb]{0.25,0.50,0.50}{##1}}}
\expandafter\def\csname PY@tok@cp\endcsname{\def\PY@tc##1{\textcolor[rgb]{0.74,0.48,0.00}{##1}}}
\expandafter\def\csname PY@tok@k\endcsname{\let\PY@bf=\textbf\def\PY@tc##1{\textcolor[rgb]{0.00,0.50,0.00}{##1}}}
\expandafter\def\csname PY@tok@kp\endcsname{\def\PY@tc##1{\textcolor[rgb]{0.00,0.50,0.00}{##1}}}
\expandafter\def\csname PY@tok@kt\endcsname{\def\PY@tc##1{\textcolor[rgb]{0.69,0.00,0.25}{##1}}}
\expandafter\def\csname PY@tok@o\endcsname{\def\PY@tc##1{\textcolor[rgb]{0.40,0.40,0.40}{##1}}}
\expandafter\def\csname PY@tok@ow\endcsname{\let\PY@bf=\textbf\def\PY@tc##1{\textcolor[rgb]{0.67,0.13,1.00}{##1}}}
\expandafter\def\csname PY@tok@nb\endcsname{\def\PY@tc##1{\textcolor[rgb]{0.00,0.50,0.00}{##1}}}
\expandafter\def\csname PY@tok@nf\endcsname{\def\PY@tc##1{\textcolor[rgb]{0.00,0.00,1.00}{##1}}}
\expandafter\def\csname PY@tok@nc\endcsname{\let\PY@bf=\textbf\def\PY@tc##1{\textcolor[rgb]{0.00,0.00,1.00}{##1}}}
\expandafter\def\csname PY@tok@nn\endcsname{\let\PY@bf=\textbf\def\PY@tc##1{\textcolor[rgb]{0.00,0.00,1.00}{##1}}}
\expandafter\def\csname PY@tok@ne\endcsname{\let\PY@bf=\textbf\def\PY@tc##1{\textcolor[rgb]{0.82,0.25,0.23}{##1}}}
\expandafter\def\csname PY@tok@nv\endcsname{\def\PY@tc##1{\textcolor[rgb]{0.10,0.09,0.49}{##1}}}
\expandafter\def\csname PY@tok@no\endcsname{\def\PY@tc##1{\textcolor[rgb]{0.53,0.00,0.00}{##1}}}
\expandafter\def\csname PY@tok@nl\endcsname{\def\PY@tc##1{\textcolor[rgb]{0.63,0.63,0.00}{##1}}}
\expandafter\def\csname PY@tok@ni\endcsname{\let\PY@bf=\textbf\def\PY@tc##1{\textcolor[rgb]{0.60,0.60,0.60}{##1}}}
\expandafter\def\csname PY@tok@na\endcsname{\def\PY@tc##1{\textcolor[rgb]{0.49,0.56,0.16}{##1}}}
\expandafter\def\csname PY@tok@nt\endcsname{\let\PY@bf=\textbf\def\PY@tc##1{\textcolor[rgb]{0.00,0.50,0.00}{##1}}}
\expandafter\def\csname PY@tok@nd\endcsname{\def\PY@tc##1{\textcolor[rgb]{0.67,0.13,1.00}{##1}}}
\expandafter\def\csname PY@tok@s\endcsname{\def\PY@tc##1{\textcolor[rgb]{0.73,0.13,0.13}{##1}}}
\expandafter\def\csname PY@tok@sd\endcsname{\let\PY@it=\textit\def\PY@tc##1{\textcolor[rgb]{0.73,0.13,0.13}{##1}}}
\expandafter\def\csname PY@tok@si\endcsname{\let\PY@bf=\textbf\def\PY@tc##1{\textcolor[rgb]{0.73,0.40,0.53}{##1}}}
\expandafter\def\csname PY@tok@se\endcsname{\let\PY@bf=\textbf\def\PY@tc##1{\textcolor[rgb]{0.73,0.40,0.13}{##1}}}
\expandafter\def\csname PY@tok@sr\endcsname{\def\PY@tc##1{\textcolor[rgb]{0.73,0.40,0.53}{##1}}}
\expandafter\def\csname PY@tok@ss\endcsname{\def\PY@tc##1{\textcolor[rgb]{0.10,0.09,0.49}{##1}}}
\expandafter\def\csname PY@tok@sx\endcsname{\def\PY@tc##1{\textcolor[rgb]{0.00,0.50,0.00}{##1}}}
\expandafter\def\csname PY@tok@m\endcsname{\def\PY@tc##1{\textcolor[rgb]{0.40,0.40,0.40}{##1}}}
\expandafter\def\csname PY@tok@gh\endcsname{\let\PY@bf=\textbf\def\PY@tc##1{\textcolor[rgb]{0.00,0.00,0.50}{##1}}}
\expandafter\def\csname PY@tok@gu\endcsname{\let\PY@bf=\textbf\def\PY@tc##1{\textcolor[rgb]{0.50,0.00,0.50}{##1}}}
\expandafter\def\csname PY@tok@gd\endcsname{\def\PY@tc##1{\textcolor[rgb]{0.63,0.00,0.00}{##1}}}
\expandafter\def\csname PY@tok@gi\endcsname{\def\PY@tc##1{\textcolor[rgb]{0.00,0.63,0.00}{##1}}}
\expandafter\def\csname PY@tok@gr\endcsname{\def\PY@tc##1{\textcolor[rgb]{1.00,0.00,0.00}{##1}}}
\expandafter\def\csname PY@tok@ge\endcsname{\let\PY@it=\textit}
\expandafter\def\csname PY@tok@gs\endcsname{\let\PY@bf=\textbf}
\expandafter\def\csname PY@tok@gp\endcsname{\let\PY@bf=\textbf\def\PY@tc##1{\textcolor[rgb]{0.00,0.00,0.50}{##1}}}
\expandafter\def\csname PY@tok@go\endcsname{\def\PY@tc##1{\textcolor[rgb]{0.53,0.53,0.53}{##1}}}
\expandafter\def\csname PY@tok@gt\endcsname{\def\PY@tc##1{\textcolor[rgb]{0.00,0.27,0.87}{##1}}}
\expandafter\def\csname PY@tok@err\endcsname{\def\PY@bc##1{\setlength{\fboxsep}{0pt}\fcolorbox[rgb]{1.00,0.00,0.00}{1,1,1}{\strut ##1}}}
\expandafter\def\csname PY@tok@kc\endcsname{\let\PY@bf=\textbf\def\PY@tc##1{\textcolor[rgb]{0.00,0.50,0.00}{##1}}}
\expandafter\def\csname PY@tok@kd\endcsname{\let\PY@bf=\textbf\def\PY@tc##1{\textcolor[rgb]{0.00,0.50,0.00}{##1}}}
\expandafter\def\csname PY@tok@kn\endcsname{\let\PY@bf=\textbf\def\PY@tc##1{\textcolor[rgb]{0.00,0.50,0.00}{##1}}}
\expandafter\def\csname PY@tok@kr\endcsname{\let\PY@bf=\textbf\def\PY@tc##1{\textcolor[rgb]{0.00,0.50,0.00}{##1}}}
\expandafter\def\csname PY@tok@bp\endcsname{\def\PY@tc##1{\textcolor[rgb]{0.00,0.50,0.00}{##1}}}
\expandafter\def\csname PY@tok@fm\endcsname{\def\PY@tc##1{\textcolor[rgb]{0.00,0.00,1.00}{##1}}}
\expandafter\def\csname PY@tok@vc\endcsname{\def\PY@tc##1{\textcolor[rgb]{0.10,0.09,0.49}{##1}}}
\expandafter\def\csname PY@tok@vg\endcsname{\def\PY@tc##1{\textcolor[rgb]{0.10,0.09,0.49}{##1}}}
\expandafter\def\csname PY@tok@vi\endcsname{\def\PY@tc##1{\textcolor[rgb]{0.10,0.09,0.49}{##1}}}
\expandafter\def\csname PY@tok@vm\endcsname{\def\PY@tc##1{\textcolor[rgb]{0.10,0.09,0.49}{##1}}}
\expandafter\def\csname PY@tok@sa\endcsname{\def\PY@tc##1{\textcolor[rgb]{0.73,0.13,0.13}{##1}}}
\expandafter\def\csname PY@tok@sb\endcsname{\def\PY@tc##1{\textcolor[rgb]{0.73,0.13,0.13}{##1}}}
\expandafter\def\csname PY@tok@sc\endcsname{\def\PY@tc##1{\textcolor[rgb]{0.73,0.13,0.13}{##1}}}
\expandafter\def\csname PY@tok@dl\endcsname{\def\PY@tc##1{\textcolor[rgb]{0.73,0.13,0.13}{##1}}}
\expandafter\def\csname PY@tok@s2\endcsname{\def\PY@tc##1{\textcolor[rgb]{0.73,0.13,0.13}{##1}}}
\expandafter\def\csname PY@tok@sh\endcsname{\def\PY@tc##1{\textcolor[rgb]{0.73,0.13,0.13}{##1}}}
\expandafter\def\csname PY@tok@s1\endcsname{\def\PY@tc##1{\textcolor[rgb]{0.73,0.13,0.13}{##1}}}
\expandafter\def\csname PY@tok@mb\endcsname{\def\PY@tc##1{\textcolor[rgb]{0.40,0.40,0.40}{##1}}}
\expandafter\def\csname PY@tok@mf\endcsname{\def\PY@tc##1{\textcolor[rgb]{0.40,0.40,0.40}{##1}}}
\expandafter\def\csname PY@tok@mh\endcsname{\def\PY@tc##1{\textcolor[rgb]{0.40,0.40,0.40}{##1}}}
\expandafter\def\csname PY@tok@mi\endcsname{\def\PY@tc##1{\textcolor[rgb]{0.40,0.40,0.40}{##1}}}
\expandafter\def\csname PY@tok@il\endcsname{\def\PY@tc##1{\textcolor[rgb]{0.40,0.40,0.40}{##1}}}
\expandafter\def\csname PY@tok@mo\endcsname{\def\PY@tc##1{\textcolor[rgb]{0.40,0.40,0.40}{##1}}}
\expandafter\def\csname PY@tok@ch\endcsname{\let\PY@it=\textit\def\PY@tc##1{\textcolor[rgb]{0.25,0.50,0.50}{##1}}}
\expandafter\def\csname PY@tok@cm\endcsname{\let\PY@it=\textit\def\PY@tc##1{\textcolor[rgb]{0.25,0.50,0.50}{##1}}}
\expandafter\def\csname PY@tok@cpf\endcsname{\let\PY@it=\textit\def\PY@tc##1{\textcolor[rgb]{0.25,0.50,0.50}{##1}}}
\expandafter\def\csname PY@tok@c1\endcsname{\let\PY@it=\textit\def\PY@tc##1{\textcolor[rgb]{0.25,0.50,0.50}{##1}}}
\expandafter\def\csname PY@tok@cs\endcsname{\let\PY@it=\textit\def\PY@tc##1{\textcolor[rgb]{0.25,0.50,0.50}{##1}}}

\def\PYZbs{\char`\\}
\def\PYZus{\char`\_}
\def\PYZob{\char`\{}
\def\PYZcb{\char`\}}
\def\PYZca{\char`\^}
\def\PYZam{\char`\&}
\def\PYZlt{\char`\<}
\def\PYZgt{\char`\>}
\def\PYZsh{\char`\#}
\def\PYZpc{\char`\%}
\def\PYZdl{\char`\$}
\def\PYZhy{\char`\-}
\def\PYZsq{\char`\'}
\def\PYZdq{\char`\"}
\def\PYZti{\char`\~}
% for compatibility with earlier versions
\def\PYZat{@}
\def\PYZlb{[}
\def\PYZrb{]}
\makeatother


    % Exact colors from NB
    \definecolor{incolor}{rgb}{0.0, 0.0, 0.5}
    \definecolor{outcolor}{rgb}{0.545, 0.0, 0.0}



    
    % Prevent overflowing lines due to hard-to-break entities
    \sloppy 
    % Setup hyperref package
    \hypersetup{
      breaklinks=true,  % so long urls are correctly broken across lines
      colorlinks=true,
      urlcolor=urlcolor,
      linkcolor=linkcolor,
      citecolor=citecolor,
      }
    % Slightly bigger margins than the latex defaults
    
    \geometry{verbose,tmargin=1in,bmargin=1in,lmargin=1in,rmargin=1in}
    
    

    \begin{document}
    
    
    \maketitle
    
    

    
    \hypertarget{snapchat-political-ads}{%
\section{Snapchat Political Ads}\label{snapchat-political-ads}}

This project uses political ads data from Snapchat, a popular social
media app. Interesting questions to consider include: - What are the
most prevalent organizations, advertisers, and ballot candidates in the
data? Do you recognize any? - What are the characteristics of ads with a
large reach, i.e., many views? What may a campaign consider when
maximizing an ad's reach? - What are the characteristics of ads with a
smaller reach, i.e., less views? Aside from funding constraints, why
might a campaign want to produce an ad with a smaller but more targeted
reach? - What are the characteristics of the most expensive ads? If a
campaign is limited on advertising funds, what type of ad may the
campaign consider? - What groups or regions are targeted frequently?
(For example, for single-gender campaigns, are men or women targeted
more frequently?) What groups or regions are targeted less frequently?
Why? Does this depend on the type of campaign? - Have the
characteristics of ads changed over time (e.g.~over the past year)? -
When is the most common local time of day for an ad's start date? What
about the most common day of week? (Make sure to account for time zones
for both questions.)

\hypertarget{getting-the-data}{%
\subsubsection{Getting the Data}\label{getting-the-data}}

The data and its corresponding data dictionary is downloadable
\href{https://www.snap.com/en-US/political-ads/}{here}. Download both
the 2018 CSV and the 2019 CSV.

The CSVs have the same filename; rename the CSVs as needed.

Note that the CSVs have the exact same columns and the exact same data
dictionaries (\texttt{readme.txt}).

\hypertarget{cleaning-and-eda}{%
\subsubsection{Cleaning and EDA}\label{cleaning-and-eda}}

\begin{itemize}
\tightlist
\item
  Concatenate the 2018 CSV and the 2019 CSV into one DataFrame so that
  we have data from both years.
\item
  Clean the data.

  \begin{itemize}
  \tightlist
  \item
    Convert \texttt{StartDate} and \texttt{EndDate} into datetime. Make
    sure the datetimes are in the correct time zone.
  \end{itemize}
\item
  Understand the data in ways relevant to your question using univariate
  and bivariate analysis of the data as well as aggregations.
\end{itemize}

\emph{Hint 1: What is the ``Z'' at the end of each timestamp?}

\emph{Hint 2: \texttt{pd.to\_datetime} will be useful here.
\texttt{Series.dt.tz\_convert} will be useful if a change in time zone
is needed.}

\emph{Tip: To visualize geospatial data, consider
\href{https://python-visualization.github.io/folium/}{Folium} or another
geospatial plotting library.}

\hypertarget{assessment-of-missingness}{%
\subsubsection{Assessment of
Missingness}\label{assessment-of-missingness}}

Many columns which have \texttt{NaN} values may not actually have
missing data. How come? In some cases, a null or empty value corresponds
to an actual, meaningful value. For example, \texttt{readme.txt} states
the following about \texttt{Gender}:

\begin{quote}
Gender - Gender targeting criteria used in the Ad. If empty, then it is
targeting all genders
\end{quote}

In this scenario, an empty \texttt{Gender} value (which is read in as
\texttt{NaN} in pandas) corresponds to ``all genders''.

\begin{itemize}
\tightlist
\item
  Refer to the data dictionary to determine which columns do
  \textbf{not} belong to the scenario above. Assess the missingness of
  one of these columns.
\end{itemize}

\hypertarget{hypothesis-test-permutation-test}{%
\subsubsection{Hypothesis Test / Permutation
Test}\label{hypothesis-test-permutation-test}}

Find a hypothesis test or permutation test to perform. You can use the
questions at the top of the notebook for inspiration.

    \hypertarget{summary-of-findings}{%
\section{Summary of Findings}\label{summary-of-findings}}

\hypertarget{introduction}{%
\subsubsection{Introduction}\label{introduction}}

The question I will be exploring is how does the amount spent on an
advertisement affect is effectiveness (measured by impressions)? The
main data I will use is the Spend and Impressions data, but the effect
of other columns on spending and impressions can also be explored.

\hypertarget{cleaning-and-eda}{%
\subsubsection{Cleaning and EDA}\label{cleaning-and-eda}}

To clean the data, I converted the StartDate and EndDate to pandas
datetime objects. I also removed any outliers in the data, and during
the hypothesis test, I filtered out ads that spent zero dollars.

In my EDA, I explored the relationship between the money spent on an ad
and the number of impressions it made. I plotted this on a linear
regression plot and found that the two variables were positively
correlated. However, upon examining the residual plot, I found that a
linear regression was not an optimal fit for the data. A complex,
customized function is likely required to model this. I then separated
the data by whether or not an target age was specified and accidentally
found that most ads that did not specify a target age also spent far
less.

\hypertarget{assessment-of-missingness}{%
\subsubsection{Assessment of
Missingness}\label{assessment-of-missingness}}

In the assessment of missingness, I checked EndDate's dependency on
StartDate. Using the permutation test with a significance level of 0.01,
I found that the missingness of data in EndDate could be reasonably
explained by the data in StartDate.

\hypertarget{hypothesis-test}{%
\subsubsection{Hypothesis Test}\label{hypothesis-test}}

I performed a hypothesis test to determine if specifying a target age
improves the number of impressions per spending.

Null Hypothesis Specifying a target age has roughly no effect on the
relationship between `Spend' and `Impressions'.

Alternative Hypothesis: Specifying a target age increases the number of
impressions per spending.

Test Statistic: The test statistic we will be using is the mean number
of impressions per dollar (USD) spent for ads that specified a target
age.

Significance Level: 0.05

The results of the hypothesis test failed to reject the null hypothesis
and had an extremely high p-value \textgreater{} .99 This suggests some
underlying problems with the set-up of the test.

    \hypertarget{code}{%
\section{Code}\label{code}}

    \begin{Verbatim}[commandchars=\\\{\}]
{\color{incolor}In [{\color{incolor}1}]:} \PY{k+kn}{import} \PY{n+nn}{matplotlib}\PY{n+nn}{.}\PY{n+nn}{pyplot} \PY{k}{as} \PY{n+nn}{plt}
        \PY{k+kn}{import} \PY{n+nn}{numpy} \PY{k}{as} \PY{n+nn}{np}
        \PY{k+kn}{import} \PY{n+nn}{os}
        \PY{k+kn}{import} \PY{n+nn}{pandas} \PY{k}{as} \PY{n+nn}{pd}
        \PY{k+kn}{import} \PY{n+nn}{seaborn} \PY{k}{as} \PY{n+nn}{sns}
        \PY{o}{\PYZpc{}}\PY{k}{matplotlib} inline
        \PY{o}{\PYZpc{}}\PY{k}{config} InlineBackend.figure\PYZus{}format = \PYZsq{}retina\PYZsq{}  \PYZsh{} Higher resolution figures
\end{Verbatim}


    \hypertarget{cleaning-and-eda}{%
\subsubsection{Cleaning and EDA}\label{cleaning-and-eda}}

    \begin{Verbatim}[commandchars=\\\{\}]
{\color{incolor}In [{\color{incolor}36}]:} \PY{c+c1}{\PYZsh{} load the csv files for 2018 and 2019}
         \PY{n}{fp\PYZus{}18} \PY{o}{=} \PY{n}{os}\PY{o}{.}\PY{n}{path}\PY{o}{.}\PY{n}{join}\PY{p}{(}\PY{l+s+s1}{\PYZsq{}}\PY{l+s+s1}{data}\PY{l+s+s1}{\PYZsq{}}\PY{p}{,} \PY{l+s+s1}{\PYZsq{}}\PY{l+s+s1}{ads\PYZus{}2018.csv}\PY{l+s+s1}{\PYZsq{}}\PY{p}{)}
         \PY{n}{fp\PYZus{}19} \PY{o}{=} \PY{n}{os}\PY{o}{.}\PY{n}{path}\PY{o}{.}\PY{n}{join}\PY{p}{(}\PY{l+s+s1}{\PYZsq{}}\PY{l+s+s1}{data}\PY{l+s+s1}{\PYZsq{}}\PY{p}{,} \PY{l+s+s1}{\PYZsq{}}\PY{l+s+s1}{ads\PYZus{}2019.csv}\PY{l+s+s1}{\PYZsq{}}\PY{p}{)}
         
         \PY{c+c1}{\PYZsh{} read files into dataframes}
         \PY{n}{df\PYZus{}18} \PY{o}{=} \PY{n}{pd}\PY{o}{.}\PY{n}{read\PYZus{}csv}\PY{p}{(}\PY{n}{fp\PYZus{}18}\PY{p}{)}
         \PY{n}{df\PYZus{}19} \PY{o}{=} \PY{n}{pd}\PY{o}{.}\PY{n}{read\PYZus{}csv}\PY{p}{(}\PY{n}{fp\PYZus{}19}\PY{p}{)}
         \PY{n}{ad\PYZus{}data} \PY{o}{=} \PY{n}{pd}\PY{o}{.}\PY{n}{concat}\PY{p}{(}\PY{p}{[}\PY{n}{df\PYZus{}18}\PY{p}{,} \PY{n}{df\PYZus{}19}\PY{p}{]}\PY{p}{)}
         \PY{n}{ad\PYZus{}data}\PY{o}{.}\PY{n}{head}\PY{p}{(}\PY{p}{)}
\end{Verbatim}


\begin{Verbatim}[commandchars=\\\{\}]
{\color{outcolor}Out[{\color{outcolor}36}]:}                                                 ADID  \textbackslash{}
         0  91db2796a80472ed8c2bfa17760b3ce1471f6ec1f3147b{\ldots}   
         1  97e3f17d5ec164c454a35d2822734482ca60be3f3af310{\ldots}   
         2  14535fea019a9b1a910a77ce1555af8bdedbb5c78fb60a{\ldots}   
         3  10b64550ad4a23c651d7883746cabeac93cbd92d5f3b3f{\ldots}   
         4  2438786c60ae41cf56614885b415a72857bbfb5c06f760{\ldots}   
         
                                                  CreativeUrl  Spend  Impressions  \textbackslash{}
         0  https://www.snap.com/political-ads/asset/b2c47{\ldots}   1044       137185   
         1  https://www.snap.com/political-ads/asset/affc7{\ldots}    279        94161   
         2  https://www.snap.com/political-ads/asset/754f6{\ldots}   6743      3149886   
         3  https://www.snap.com/political-ads/asset/818ae{\ldots}   3698       573475   
         4  https://www.snap.com/political-ads/asset/2c264{\ldots}    445       232906   
         
                       StartDate               EndDate          OrganizationName  \textbackslash{}
         0  2018/10/30 17:45:51Z  2018/11/07 00:00:00Z                 GMMB, Inc   
         1  2018/12/23 14:26:52Z  2018/12/28 14:28:06Z      Revolution Messaging   
         2  2018/10/06 01:11:41Z  2018/11/07 03:00:00Z         Lockwood Strategy   
         3  2018/11/02 16:20:57Z  2018/11/06 18:15:30Z         The Prosper Group   
         4  2018/11/27 21:44:19Z  2019/01/13 21:43:53Z  Amnesty International UK   
         
                                          BillingAddress CandidateBallotInformation  \textbackslash{}
         0             3050 K Street,Washington,20007,US                        NaN   
         1  1730 Rhode Island Ave NW,Washington,20036,US                        NaN   
         2                                            US                        NaN   
         3                435 E. Main,Greenwood,46143,US                        NaN   
         4         17-25 New Inn Yard,London,EC2A 3EA,GB                        NaN   
         
                PayingAdvertiserName  \textbackslash{}
         0           JB for Governor   
         1   Paid for by ReBuild USA   
         2                Change Now   
         3                   No On L   
         4  Amnesty International UK   
         
                                  {\ldots}                          \textbackslash{}
         0                        {\ldots}                           
         1                        {\ldots}                           
         2                        {\ldots}                           
         3                        {\ldots}                           
         4                        {\ldots}                           
         
                                                    Interests OsType  \textbackslash{}
         0                                                NaN    NaN   
         1  Arts \& Culture Mavens,Chat Fiction Enthusiasts{\ldots}    NaN   
         2  TV Live Event Viewers (The Academy Awards),TV {\ldots}    NaN   
         3                                                NaN    NaN   
         4                                                NaN    NaN   
         
                          Segments LocationType Language  AdvancedDemographics  \textbackslash{}
         0  Provided by Advertiser          NaN      NaN                   NaN   
         1  Provided by Advertiser          NaN      NaN                   NaN   
         2  Provided by Advertiser          NaN      NaN                   NaN   
         3  Provided by Advertiser          NaN      NaN                   NaN   
         4  Provided by Advertiser          NaN      NaN                   NaN   
         
           Targeting Connection Type Targeting Carrier (ISP)  \textbackslash{}
         0                       NaN                     NaN   
         1                       NaN                     NaN   
         2                       NaN                     NaN   
         3                       NaN                     NaN   
         4                       NaN                     NaN   
         
                                  Targeting Geo - Postal Code  \textbackslash{}
         0                                                NaN   
         1                                                NaN   
         2                                                NaN   
         3  92801,92802,92803,92804,92805,92806,92807,9280{\ldots}   
         4                                                NaN   
         
                                           CreativeProperties  
         0       web\_view\_url:https://iwillvote.com/?state=il  
         1     web\_view\_url:https://rebuildusa.info/landing-3  
         2  web\_view\_url:https://action.socalhealthcarecoa{\ldots}  
         3          web\_view\_url:https://www.stopmeasurel.com  
         4  web\_view\_url:https://www.amnesty.org.uk/write-{\ldots}  
         
         [5 rows x 27 columns]
\end{Verbatim}
            
    \begin{Verbatim}[commandchars=\\\{\}]
{\color{incolor}In [{\color{incolor}37}]:} \PY{c+c1}{\PYZsh{} convert StartDate and EndDate to date time objects (UTC)}
         \PY{c+c1}{\PYZsh{} timezones will not be used in our analysis}
         \PY{n}{ad\PYZus{}data}\PY{p}{[}\PY{l+s+s1}{\PYZsq{}}\PY{l+s+s1}{StartDate}\PY{l+s+s1}{\PYZsq{}}\PY{p}{]} \PY{o}{=} \PY{n}{pd}\PY{o}{.}\PY{n}{to\PYZus{}datetime}\PY{p}{(}\PY{n}{ad\PYZus{}data}\PY{p}{[}\PY{l+s+s1}{\PYZsq{}}\PY{l+s+s1}{StartDate}\PY{l+s+s1}{\PYZsq{}}\PY{p}{]}\PY{p}{)}
         \PY{n}{ad\PYZus{}data}\PY{p}{[}\PY{l+s+s1}{\PYZsq{}}\PY{l+s+s1}{EndDate}\PY{l+s+s1}{\PYZsq{}}\PY{p}{]} \PY{o}{=} \PY{n}{pd}\PY{o}{.}\PY{n}{to\PYZus{}datetime}\PY{p}{(}\PY{n}{ad\PYZus{}data}\PY{p}{[}\PY{l+s+s1}{\PYZsq{}}\PY{l+s+s1}{EndDate}\PY{l+s+s1}{\PYZsq{}}\PY{p}{]}\PY{p}{)}
         
         \PY{c+c1}{\PYZsh{} display the data type of each column}
         \PY{n}{ad\PYZus{}data}\PY{o}{.}\PY{n}{info}\PY{p}{(}\PY{p}{)}
\end{Verbatim}


    \begin{Verbatim}[commandchars=\\\{\}]
<class 'pandas.core.frame.DataFrame'>
Int64Index: 3303 entries, 0 to 2643
Data columns (total 27 columns):
ADID                           3303 non-null object
CreativeUrl                    3303 non-null object
Spend                          3303 non-null int64
Impressions                    3303 non-null int64
StartDate                      3303 non-null datetime64[ns]
EndDate                        2647 non-null datetime64[ns]
OrganizationName               3303 non-null object
BillingAddress                 3303 non-null object
CandidateBallotInformation     225 non-null object
PayingAdvertiserName           3303 non-null object
Gender                         322 non-null object
AgeBracket                     3029 non-null object
CountryCode                    3303 non-null object
RegionID                       1013 non-null object
ElectoralDistrictID            65 non-null object
LatLongRad                     0 non-null float64
MetroID                        180 non-null object
Interests                      786 non-null object
OsType                         21 non-null object
Segments                       2189 non-null object
LocationType                   18 non-null object
Language                       914 non-null object
AdvancedDemographics           96 non-null object
Targeting Connection Type      0 non-null float64
Targeting Carrier (ISP)        0 non-null float64
Targeting Geo - Postal Code    399 non-null object
CreativeProperties             2656 non-null object
dtypes: datetime64[ns](2), float64(3), int64(2), object(20)
memory usage: 722.5+ KB

    \end{Verbatim}

    \begin{Verbatim}[commandchars=\\\{\}]
{\color{incolor}In [{\color{incolor}4}]:} \PY{c+c1}{\PYZsh{} Exploratory Data Analysis}
        
        \PY{c+c1}{\PYZsh{} we will begin by examining how much is spent on ads}
        
        \PY{c+c1}{\PYZsh{} plot the spending data }
        \PY{c+c1}{\PYZsh{} plot types: hist, kde, rug}
        \PY{n}{sns}\PY{o}{.}\PY{n}{distplot}\PY{p}{(}\PY{n}{ad\PYZus{}data}\PY{p}{[}\PY{l+s+s1}{\PYZsq{}}\PY{l+s+s1}{Spend}\PY{l+s+s1}{\PYZsq{}}\PY{p}{]}\PY{p}{,} \PY{n}{hist}\PY{o}{=}\PY{k+kc}{True}\PY{p}{,} \PY{n}{kde}\PY{o}{=}\PY{k+kc}{True}\PY{p}{,} \PY{n}{rug}\PY{o}{=}\PY{k+kc}{True}\PY{p}{)}
\end{Verbatim}


\begin{Verbatim}[commandchars=\\\{\}]
{\color{outcolor}Out[{\color{outcolor}4}]:} <matplotlib.axes.\_subplots.AxesSubplot at 0x1e27649aa90>
\end{Verbatim}
            
    \begin{center}
    \adjustimage{max size={0.9\linewidth}{0.9\paperheight}}{output_7_1.png}
    \end{center}
    { \hspace*{\fill} \\}
    
    \begin{Verbatim}[commandchars=\\\{\}]
{\color{incolor}In [{\color{incolor}5}]:} \PY{c+c1}{\PYZsh{} next, we look at the effectiveness of those ads in the form of impressions}
        
        \PY{c+c1}{\PYZsh{} plot the impressions data }
        \PY{c+c1}{\PYZsh{} plot types: hist, kde, rug}
        \PY{n}{sns}\PY{o}{.}\PY{n}{distplot}\PY{p}{(}\PY{n}{ad\PYZus{}data}\PY{p}{[}\PY{l+s+s1}{\PYZsq{}}\PY{l+s+s1}{Impressions}\PY{l+s+s1}{\PYZsq{}}\PY{p}{]}\PY{p}{,} \PY{n}{hist}\PY{o}{=}\PY{k+kc}{True}\PY{p}{,} \PY{n}{kde}\PY{o}{=}\PY{k+kc}{True}\PY{p}{,} \PY{n}{rug}\PY{o}{=}\PY{k+kc}{True}\PY{p}{)}
\end{Verbatim}


\begin{Verbatim}[commandchars=\\\{\}]
{\color{outcolor}Out[{\color{outcolor}5}]:} <matplotlib.axes.\_subplots.AxesSubplot at 0x1e276918b00>
\end{Verbatim}
            
    \begin{center}
    \adjustimage{max size={0.9\linewidth}{0.9\paperheight}}{output_8_1.png}
    \end{center}
    { \hspace*{\fill} \\}
    
    \begin{Verbatim}[commandchars=\\\{\}]
{\color{incolor}In [{\color{incolor}40}]:} \PY{c+c1}{\PYZsh{} looking at the plots, we see that both plots have very large outliers.}
         \PY{c+c1}{\PYZsh{} the existance those outliers makes the above plots rather meaningless,}
         \PY{c+c1}{\PYZsh{} so we will now remove those outliers}
         
         \PY{c+c1}{\PYZsh{} we define an outlier to be any datapoint greater than 3 standard deviations}
         \PY{c+c1}{\PYZsh{} away from the mean}
         \PY{n}{spend\PYZus{}mean} \PY{o}{=} \PY{n}{ad\PYZus{}data}\PY{p}{[}\PY{l+s+s1}{\PYZsq{}}\PY{l+s+s1}{Spend}\PY{l+s+s1}{\PYZsq{}}\PY{p}{]}\PY{o}{.}\PY{n}{mean}\PY{p}{(}\PY{p}{)}
         \PY{n}{spend\PYZus{}std} \PY{o}{=} \PY{n}{ad\PYZus{}data}\PY{p}{[}\PY{l+s+s1}{\PYZsq{}}\PY{l+s+s1}{Spend}\PY{l+s+s1}{\PYZsq{}}\PY{p}{]}\PY{o}{.}\PY{n}{std}\PY{p}{(}\PY{p}{)}
         \PY{n}{spend\PYZus{}no\PYZus{}out} \PY{o}{=} \PY{n}{ad\PYZus{}data}\PY{p}{[}\PY{l+s+s1}{\PYZsq{}}\PY{l+s+s1}{Spend}\PY{l+s+s1}{\PYZsq{}}\PY{p}{]}\PY{p}{[}\PY{n+nb}{abs}\PY{p}{(}\PY{n}{spend\PYZus{}mean} \PY{o}{\PYZhy{}} \PY{n}{ad\PYZus{}data}\PY{p}{[}\PY{l+s+s1}{\PYZsq{}}\PY{l+s+s1}{Spend}\PY{l+s+s1}{\PYZsq{}}\PY{p}{]}\PY{p}{)}\PY{o}{/}\PY{n}{spend\PYZus{}std} \PY{o}{\PYZlt{}}\PY{o}{=} \PY{l+m+mi}{3}\PY{p}{]}
         
         
         \PY{c+c1}{\PYZsh{} plot the spend data for with outliers removed}
         \PY{c+c1}{\PYZsh{} plot types: hist, kde, rug}
         \PY{n}{sns}\PY{o}{.}\PY{n}{distplot}\PY{p}{(}\PY{n}{spend\PYZus{}no\PYZus{}out}\PY{p}{,} \PY{n}{hist}\PY{o}{=}\PY{k+kc}{True}\PY{p}{,} \PY{n}{kde}\PY{o}{=}\PY{k+kc}{True}\PY{p}{,} \PY{n}{rug}\PY{o}{=}\PY{k+kc}{True}\PY{p}{)}
\end{Verbatim}


\begin{Verbatim}[commandchars=\\\{\}]
{\color{outcolor}Out[{\color{outcolor}40}]:} <matplotlib.axes.\_subplots.AxesSubplot at 0x1e277d0c898>
\end{Verbatim}
            
    \begin{center}
    \adjustimage{max size={0.9\linewidth}{0.9\paperheight}}{output_9_1.png}
    \end{center}
    { \hspace*{\fill} \\}
    
    \begin{Verbatim}[commandchars=\\\{\}]
{\color{incolor}In [{\color{incolor}42}]:} \PY{c+c1}{\PYZsh{} repeat the process for impressions}
         
         \PY{n}{imp\PYZus{}mean} \PY{o}{=} \PY{n}{ad\PYZus{}data}\PY{p}{[}\PY{l+s+s1}{\PYZsq{}}\PY{l+s+s1}{Impressions}\PY{l+s+s1}{\PYZsq{}}\PY{p}{]}\PY{o}{.}\PY{n}{mean}\PY{p}{(}\PY{p}{)}
         \PY{n}{imp\PYZus{}std} \PY{o}{=} \PY{n}{ad\PYZus{}data}\PY{p}{[}\PY{l+s+s1}{\PYZsq{}}\PY{l+s+s1}{Impressions}\PY{l+s+s1}{\PYZsq{}}\PY{p}{]}\PY{o}{.}\PY{n}{std}\PY{p}{(}\PY{p}{)}
         \PY{n}{imp\PYZus{}no\PYZus{}out} \PY{o}{=} \PY{n}{ad\PYZus{}data}\PY{p}{[}\PY{l+s+s1}{\PYZsq{}}\PY{l+s+s1}{Impressions}\PY{l+s+s1}{\PYZsq{}}\PY{p}{]}\PY{p}{[}\PY{n+nb}{abs}\PY{p}{(}\PY{n}{imp\PYZus{}mean} \PY{o}{\PYZhy{}} \PY{n}{ad\PYZus{}data}\PY{p}{[}\PY{l+s+s1}{\PYZsq{}}\PY{l+s+s1}{Impressions}\PY{l+s+s1}{\PYZsq{}}\PY{p}{]}\PY{p}{)}\PY{o}{/}\PY{n}{imp\PYZus{}std} \PY{o}{\PYZlt{}}\PY{o}{=} \PY{l+m+mi}{3}\PY{p}{]}
         
         \PY{c+c1}{\PYZsh{} plot the impressions data for with outliers removed}
         \PY{c+c1}{\PYZsh{} plot types: hist, kde, rug}
         \PY{n}{sns}\PY{o}{.}\PY{n}{distplot}\PY{p}{(}\PY{n}{imp\PYZus{}no\PYZus{}out}\PY{p}{,} \PY{n}{hist}\PY{o}{=}\PY{k+kc}{True}\PY{p}{,} \PY{n}{kde}\PY{o}{=}\PY{k+kc}{True}\PY{p}{,} \PY{n}{rug}\PY{o}{=}\PY{k+kc}{True}\PY{p}{)}
         
         \PY{n}{sns}\PY{o}{.}\PY{n}{distplot}\PY{p}{(}\PY{n}{spend\PYZus{}no\PYZus{}out}\PY{p}{,} \PY{n}{hist}\PY{o}{=}\PY{k+kc}{True}\PY{p}{,} \PY{n}{kde}\PY{o}{=}\PY{k+kc}{True}\PY{p}{,} \PY{n}{rug}\PY{o}{=}\PY{k+kc}{True}\PY{p}{)}
\end{Verbatim}


\begin{Verbatim}[commandchars=\\\{\}]
{\color{outcolor}Out[{\color{outcolor}42}]:} <matplotlib.axes.\_subplots.AxesSubplot at 0x1e27c9ddcf8>
\end{Verbatim}
            
    \begin{center}
    \adjustimage{max size={0.9\linewidth}{0.9\paperheight}}{output_10_1.png}
    \end{center}
    { \hspace*{\fill} \\}
    
    \begin{Verbatim}[commandchars=\\\{\}]
{\color{incolor}In [{\color{incolor}87}]:} \PY{c+c1}{\PYZsh{} while the distributions still look similar, the plots are significantly different}
         
         \PY{c+c1}{\PYZsh{} create variables of spend and imp for convenience}
         \PY{n}{spend} \PY{o}{=} \PY{n}{ad\PYZus{}data}\PY{p}{[}\PY{l+s+s1}{\PYZsq{}}\PY{l+s+s1}{Spend}\PY{l+s+s1}{\PYZsq{}}\PY{p}{]}
         \PY{n}{imp} \PY{o}{=} \PY{n}{ad\PYZus{}data}\PY{p}{[}\PY{l+s+s1}{\PYZsq{}}\PY{l+s+s1}{Impressions}\PY{l+s+s1}{\PYZsq{}}\PY{p}{]}
         
         \PY{c+c1}{\PYZsh{} calculate the statistics for each data series using pandas\PYZsq{} Series.describe()}
         \PY{c+c1}{\PYZsh{} rename the series without outliers to clarify}
         \PY{n}{spend\PYZus{}stats} \PY{o}{=} \PY{n}{ad\PYZus{}data}\PY{p}{[}\PY{l+s+s1}{\PYZsq{}}\PY{l+s+s1}{Spend}\PY{l+s+s1}{\PYZsq{}}\PY{p}{]}\PY{o}{.}\PY{n}{describe}\PY{p}{(}\PY{p}{)}
         \PY{n}{spend\PYZus{}no\PYZus{}out\PYZus{}stats} \PY{o}{=} \PY{n}{spend\PYZus{}no\PYZus{}out}\PY{o}{.}\PY{n}{describe}\PY{p}{(}\PY{p}{)}\PY{o}{.}\PY{n}{rename}\PY{p}{(}\PY{l+s+s1}{\PYZsq{}}\PY{l+s+s1}{Spend no outliers}\PY{l+s+s1}{\PYZsq{}}\PY{p}{)}
         \PY{n}{imp\PYZus{}stats} \PY{o}{=} \PY{n}{ad\PYZus{}data}\PY{p}{[}\PY{l+s+s1}{\PYZsq{}}\PY{l+s+s1}{Impressions}\PY{l+s+s1}{\PYZsq{}}\PY{p}{]}\PY{o}{.}\PY{n}{describe}\PY{p}{(}\PY{p}{)}
         \PY{n}{imp\PYZus{}no\PYZus{}out\PYZus{}stats} \PY{o}{=} \PY{n}{imp\PYZus{}no\PYZus{}out}\PY{o}{.}\PY{n}{describe}\PY{p}{(}\PY{p}{)}\PY{o}{.}\PY{n}{rename}\PY{p}{(}\PY{l+s+s1}{\PYZsq{}}\PY{l+s+s1}{Impressions no outliers}\PY{l+s+s1}{\PYZsq{}}\PY{p}{)}
         
         \PY{c+c1}{\PYZsh{} display summary stats in data frame}
         \PY{n}{pd}\PY{o}{.}\PY{n}{DataFrame}\PY{p}{(}\PY{p}{[}
             \PY{n}{spend\PYZus{}stats}\PY{p}{,} 
             \PY{n}{spend\PYZus{}no\PYZus{}out\PYZus{}stats}\PY{p}{,} 
             \PY{n}{imp\PYZus{}stats}\PY{p}{,} 
             \PY{n}{imp\PYZus{}no\PYZus{}out\PYZus{}stats}
         \PY{p}{]}\PY{p}{)}
\end{Verbatim}


\begin{Verbatim}[commandchars=\\\{\}]
{\color{outcolor}Out[{\color{outcolor}87}]:}                           count           mean           std  min      25\%  \textbackslash{}
         Spend                    3303.0    1564.264608  9.679614e+03  0.0     41.0   
         Spend no outliers        3282.0    1005.408592  2.626486e+03  0.0     41.0   
         Impressions              3303.0  670984.530427  3.949157e+06  1.0  13468.0   
         Impressions no outliers  3277.0  422870.233750  1.133705e+06  1.0  13333.0   
         
                                      50\%        75\%          max  
         Spend                      184.0     797.50     319467.0  
         Spend no outliers          178.0     774.75      29215.0  
         Impressions              68914.0  316056.50  150532010.0  
         Impressions no outliers  67449.0  304845.00   12414920.0  
\end{Verbatim}
            
    \begin{Verbatim}[commandchars=\\\{\}]
{\color{incolor}In [{\color{incolor}118}]:} \PY{c+c1}{\PYZsh{} looking at the counts column, we see that we retained most of the data}
          
          
          \PY{c+c1}{\PYZsh{} we will now plot spend and impressions against each other}
          
          \PY{c+c1}{\PYZsh{} get a dataframe with outliers in spend and impressions removed}
          \PY{n}{data\PYZus{}no\PYZus{}out} \PY{o}{=} \PY{n}{ad\PYZus{}data}\PY{p}{[}
              \PY{p}{(}\PY{n+nb}{abs}\PY{p}{(}\PY{n}{spend\PYZus{}mean} \PY{o}{\PYZhy{}} \PY{n}{ad\PYZus{}data}\PY{p}{[}\PY{l+s+s1}{\PYZsq{}}\PY{l+s+s1}{Spend}\PY{l+s+s1}{\PYZsq{}}\PY{p}{]}\PY{p}{)}\PY{o}{/}\PY{n}{spend\PYZus{}std} \PY{o}{\PYZlt{}}\PY{o}{=} \PY{l+m+mi}{3}\PY{p}{)} \PY{o}{\PYZam{}} 
              \PY{p}{(}\PY{n+nb}{abs}\PY{p}{(}\PY{n}{imp\PYZus{}mean} \PY{o}{\PYZhy{}} \PY{n}{ad\PYZus{}data}\PY{p}{[}\PY{l+s+s1}{\PYZsq{}}\PY{l+s+s1}{Impressions}\PY{l+s+s1}{\PYZsq{}}\PY{p}{]}\PY{p}{)}\PY{o}{/}\PY{n}{imp\PYZus{}std} \PY{o}{\PYZlt{}}\PY{o}{=} \PY{l+m+mi}{3}\PY{p}{)}
          \PY{p}{]}
          
          \PY{c+c1}{\PYZsh{} plot the linear regression plot for Spend and Impressions}
          \PY{n}{sns}\PY{o}{.}\PY{n}{lmplot}\PY{p}{(}\PY{n}{data}\PY{o}{=}\PY{n}{data\PYZus{}no\PYZus{}out}\PY{p}{,} \PY{n}{x}\PY{o}{=}\PY{l+s+s1}{\PYZsq{}}\PY{l+s+s1}{Spend}\PY{l+s+s1}{\PYZsq{}}\PY{p}{,} \PY{n}{y}\PY{o}{=}\PY{l+s+s1}{\PYZsq{}}\PY{l+s+s1}{Impressions}\PY{l+s+s1}{\PYZsq{}}\PY{p}{)}
\end{Verbatim}


\begin{Verbatim}[commandchars=\\\{\}]
{\color{outcolor}Out[{\color{outcolor}118}]:} <seaborn.axisgrid.FacetGrid at 0x1e2079b4ef0>
\end{Verbatim}
            
    \begin{center}
    \adjustimage{max size={0.9\linewidth}{0.9\paperheight}}{output_12_1.png}
    \end{center}
    { \hspace*{\fill} \\}
    
    \begin{Verbatim}[commandchars=\\\{\}]
{\color{incolor}In [{\color{incolor}119}]:} \PY{c+c1}{\PYZsh{} to check how well this regression line fits, we can look at the residual plot}
          \PY{n}{sns}\PY{o}{.}\PY{n}{residplot}\PY{p}{(}\PY{n}{data}\PY{o}{=}\PY{n}{data\PYZus{}no\PYZus{}out}\PY{p}{,} \PY{n}{x}\PY{o}{=}\PY{l+s+s1}{\PYZsq{}}\PY{l+s+s1}{Spend}\PY{l+s+s1}{\PYZsq{}}\PY{p}{,} \PY{n}{y}\PY{o}{=}\PY{l+s+s1}{\PYZsq{}}\PY{l+s+s1}{Impressions}\PY{l+s+s1}{\PYZsq{}}\PY{p}{)}
\end{Verbatim}


\begin{Verbatim}[commandchars=\\\{\}]
{\color{outcolor}Out[{\color{outcolor}119}]:} <matplotlib.axes.\_subplots.AxesSubplot at 0x1e207a0e860>
\end{Verbatim}
            
    \begin{center}
    \adjustimage{max size={0.9\linewidth}{0.9\paperheight}}{output_13_1.png}
    \end{center}
    { \hspace*{\fill} \\}
    
    \begin{Verbatim}[commandchars=\\\{\}]
{\color{incolor}In [{\color{incolor}123}]:} \PY{c+c1}{\PYZsh{} the data is not scattered above and below the regression line randomly; however, }
          \PY{c+c1}{\PYZsh{} modelling this data accurately appears to require a complex, customized function}
          
          
          \PY{c+c1}{\PYZsh{} we can compare these residuals to the residuals we would have had we not removed the outliers}
          \PY{n}{sns}\PY{o}{.}\PY{n}{residplot}\PY{p}{(}\PY{n}{data}\PY{o}{=}\PY{n}{ad\PYZus{}data}\PY{p}{,} \PY{n}{x}\PY{o}{=}\PY{l+s+s1}{\PYZsq{}}\PY{l+s+s1}{Spend}\PY{l+s+s1}{\PYZsq{}}\PY{p}{,} \PY{n}{y}\PY{o}{=}\PY{l+s+s1}{\PYZsq{}}\PY{l+s+s1}{Impressions}\PY{l+s+s1}{\PYZsq{}}\PY{p}{)}
\end{Verbatim}


\begin{Verbatim}[commandchars=\\\{\}]
{\color{outcolor}Out[{\color{outcolor}123}]:} <matplotlib.axes.\_subplots.AxesSubplot at 0x1e209437438>
\end{Verbatim}
            
    \begin{center}
    \adjustimage{max size={0.9\linewidth}{0.9\paperheight}}{output_14_1.png}
    \end{center}
    { \hspace*{\fill} \\}
    
    It seems as if the removal of outliers did in fact improve the
usefulness of our plots/visualizations.

We can also look at how other variables affect the relationship between
Spend and Impressions; we can split the data into ads that target an age
demographic (non null values) and ads that do not target an age
demographic (null values).

    \begin{Verbatim}[commandchars=\\\{\}]
{\color{incolor}In [{\color{incolor}125}]:} \PY{c+c1}{\PYZsh{} create a new data frame with relevant columns}
          \PY{n}{spend\PYZus{}imp\PYZus{}by\PYZus{}age} \PY{o}{=} \PY{n}{data\PYZus{}no\PYZus{}out}\PY{p}{[}\PY{p}{[}\PY{l+s+s1}{\PYZsq{}}\PY{l+s+s1}{Spend}\PY{l+s+s1}{\PYZsq{}}\PY{p}{,} \PY{l+s+s1}{\PYZsq{}}\PY{l+s+s1}{Impressions}\PY{l+s+s1}{\PYZsq{}}\PY{p}{,} \PY{l+s+s1}{\PYZsq{}}\PY{l+s+s1}{AgeBracket}\PY{l+s+s1}{\PYZsq{}}\PY{p}{]}\PY{p}{]}\PY{o}{.}\PY{n}{copy}\PY{p}{(}\PY{p}{)}
          
          \PY{c+c1}{\PYZsh{} add a column \PYZsq{}NullAge\PYZsq{} that is True if the ad has a null value in \PYZsq{}AgeBracket\PYZsq{} and False otherwise}
          \PY{n}{spend\PYZus{}imp\PYZus{}by\PYZus{}age}\PY{p}{[}\PY{l+s+s1}{\PYZsq{}}\PY{l+s+s1}{NullAge}\PY{l+s+s1}{\PYZsq{}}\PY{p}{]} \PY{o}{=} \PY{n}{spend\PYZus{}imp\PYZus{}by\PYZus{}age}\PY{p}{[}\PY{l+s+s1}{\PYZsq{}}\PY{l+s+s1}{AgeBracket}\PY{l+s+s1}{\PYZsq{}}\PY{p}{]}\PY{o}{.}\PY{n}{isna}\PY{p}{(}\PY{p}{)}
          
          \PY{c+c1}{\PYZsh{} plot the linear regression plots separated by \PYZsq{}NullAge\PYZsq{}}
          \PY{n}{sns}\PY{o}{.}\PY{n}{lmplot}\PY{p}{(}\PY{n}{data}\PY{o}{=}\PY{n}{spend\PYZus{}imp\PYZus{}by\PYZus{}age}\PY{p}{,} \PY{n}{x}\PY{o}{=}\PY{l+s+s1}{\PYZsq{}}\PY{l+s+s1}{Spend}\PY{l+s+s1}{\PYZsq{}}\PY{p}{,} \PY{n}{y}\PY{o}{=}\PY{l+s+s1}{\PYZsq{}}\PY{l+s+s1}{Impressions}\PY{l+s+s1}{\PYZsq{}}\PY{p}{,} \PY{n}{hue}\PY{o}{=}\PY{l+s+s1}{\PYZsq{}}\PY{l+s+s1}{NullAge}\PY{l+s+s1}{\PYZsq{}}\PY{p}{)}
\end{Verbatim}


\begin{Verbatim}[commandchars=\\\{\}]
{\color{outcolor}Out[{\color{outcolor}125}]:} <seaborn.axisgrid.FacetGrid at 0x1e20953acf8>
\end{Verbatim}
            
    \begin{center}
    \adjustimage{max size={0.9\linewidth}{0.9\paperheight}}{output_16_1.png}
    \end{center}
    { \hspace*{\fill} \\}
    
    Based on the plot, it seems as if targeting age has roughly no effect on
relationship between `Spend' and `Impressions'. However, a notable
feature of this plot is that the majority of the ads that did not
specify a target age group spent relatively little (\textless{}\$5000)
on the ad. A possible explanation for this is that Snapchat charges more
for ads that specify certain demographics. Another explanation could be
that only larger companies and organizations have the need or prior data
to specify a target demographic. Further reserach and analysis is
required to determine the reason for this.

    \hypertarget{assessment-of-missingness}{%
\subsubsection{Assessment of
Missingness}\label{assessment-of-missingness}}

    To start, we will find the proportion of null values for each column.

    \begin{Verbatim}[commandchars=\\\{\}]
{\color{incolor}In [{\color{incolor}133}]:} \PY{c+c1}{\PYZsh{} get the proportion of null values for each column}
          \PY{n}{prop\PYZus{}null} \PY{o}{=} \PY{n}{ad\PYZus{}data}\PY{o}{.}\PY{n}{isna}\PY{p}{(}\PY{p}{)}\PY{o}{.}\PY{n}{mean}\PY{p}{(}\PY{p}{)}\PY{c+c1}{\PYZsh{} \PYZgt{} 0}
          \PY{c+c1}{\PYZsh{} find the columns with null values}
          \PY{n}{has\PYZus{}null} \PY{o}{=} \PY{n}{pd}\PY{o}{.}\PY{n}{Series}\PY{p}{(}\PY{n}{prop\PYZus{}null}\PY{p}{[}\PY{n}{prop\PYZus{}null} \PY{o}{\PYZgt{}} \PY{l+m+mi}{0}\PY{p}{]}\PY{o}{.}\PY{n}{index}\PY{p}{)}
          
          \PY{c+c1}{\PYZsh{} the proportion of null values in each column}
          \PY{n}{prop\PYZus{}null}
\end{Verbatim}


\begin{Verbatim}[commandchars=\\\{\}]
{\color{outcolor}Out[{\color{outcolor}133}]:} ADID                           0.000000
          CreativeUrl                    0.000000
          Spend                          0.000000
          Impressions                    0.000000
          StartDate                      0.000000
          EndDate                        0.198607
          OrganizationName               0.000000
          BillingAddress                 0.000000
          CandidateBallotInformation     0.931880
          PayingAdvertiserName           0.000000
          Gender                         0.902513
          AgeBracket                     0.082955
          CountryCode                    0.000000
          RegionID                       0.693309
          ElectoralDistrictID            0.980321
          LatLongRad                     1.000000
          MetroID                        0.945504
          Interests                      0.762035
          OsType                         0.993642
          Segments                       0.337269
          LocationType                   0.994550
          Language                       0.723282
          AdvancedDemographics           0.970936
          Targeting Connection Type      1.000000
          Targeting Carrier (ISP)        1.000000
          Targeting Geo - Postal Code    0.879201
          CreativeProperties             0.195883
          dtype: float64
\end{Verbatim}
            
    \begin{Verbatim}[commandchars=\\\{\}]
{\color{incolor}In [{\color{incolor}129}]:} \PY{c+c1}{\PYZsh{} the following columns have null values}
          \PY{n+nb}{print}\PY{p}{(}\PY{n}{has\PYZus{}null}\PY{p}{)}
\end{Verbatim}


    \begin{Verbatim}[commandchars=\\\{\}]
0                         EndDate
1      CandidateBallotInformation
2                          Gender
3                      AgeBracket
4                        RegionID
5             ElectoralDistrictID
6                      LatLongRad
7                         MetroID
8                       Interests
9                          OsType
10                       Segments
11                   LocationType
12                       Language
13           AdvancedDemographics
14      Targeting Connection Type
15        Targeting Carrier (ISP)
16    Targeting Geo - Postal Code
17             CreativeProperties
dtype: object

    \end{Verbatim}

    Based on the readme.txt provided with the data, the following columns
have values missing by design:

\begin{itemize}
\tightlist
\item
  \textbf{Gender:} if null, then target all genders
\item
  \textbf{AgeBracket:} if null, then target all ages
\item
  \textbf{RegionID:} if null, then target all regions in target country
\item
  \textbf{ElectoralDistrickID:} if null, then target all electoral
  districts in the target country
\item
  \textbf{LatLongRad:} if null, then target all lat/long in the target
  country
\item
  \textbf{MetroID:} if null, then target all metros in the target
  country
\item
  \textbf{Interests:} if null, then the ad is agnostic to interests
\item
  \textbf{OsType:} if null, then target all operating systems
\item
  \textbf{Language:} if null, then the ad is agnostic to language
\item
  \textbf{AdvancedDemographics:} if null, then the ad is agnostic to 3rd
  party data segments
\item
  \textbf{Targeting Connection Type:} if null, then the ad is agnostic
  to internet connection type
\item
  \textbf{Targeting Carrier (ISP):} if null, then the ad is agnostic to
  carrier type
\item
  \textbf{Targeting Geo-Postal Code:} if null, then targets all postal
  codes in the target country
\end{itemize}

    \begin{Verbatim}[commandchars=\\\{\}]
{\color{incolor}In [{\color{incolor}135}]:} \PY{c+c1}{\PYZsh{} columns missing by design}
          \PY{n}{mbd} \PY{o}{=} \PY{p}{[}
              \PY{l+s+s1}{\PYZsq{}}\PY{l+s+s1}{Gender}\PY{l+s+s1}{\PYZsq{}}\PY{p}{,} \PY{l+s+s1}{\PYZsq{}}\PY{l+s+s1}{AgeBracket}\PY{l+s+s1}{\PYZsq{}}\PY{p}{,} \PY{l+s+s1}{\PYZsq{}}\PY{l+s+s1}{RegionID}\PY{l+s+s1}{\PYZsq{}}\PY{p}{,} \PY{l+s+s1}{\PYZsq{}}\PY{l+s+s1}{ElectoralDistrictID}\PY{l+s+s1}{\PYZsq{}}\PY{p}{,} \PY{l+s+s1}{\PYZsq{}}\PY{l+s+s1}{LatLongRad}\PY{l+s+s1}{\PYZsq{}}\PY{p}{,} \PY{l+s+s1}{\PYZsq{}}\PY{l+s+s1}{MetroID}\PY{l+s+s1}{\PYZsq{}}\PY{p}{,}
              \PY{l+s+s1}{\PYZsq{}}\PY{l+s+s1}{Interests}\PY{l+s+s1}{\PYZsq{}}\PY{p}{,} \PY{l+s+s1}{\PYZsq{}}\PY{l+s+s1}{OsType}\PY{l+s+s1}{\PYZsq{}}\PY{p}{,} \PY{l+s+s1}{\PYZsq{}}\PY{l+s+s1}{Language}\PY{l+s+s1}{\PYZsq{}}\PY{p}{,} \PY{l+s+s1}{\PYZsq{}}\PY{l+s+s1}{AdvancedDemographics}\PY{l+s+s1}{\PYZsq{}}\PY{p}{,} \PY{l+s+s1}{\PYZsq{}}\PY{l+s+s1}{Targeting Connection Type}\PY{l+s+s1}{\PYZsq{}}\PY{p}{,} 
              \PY{l+s+s1}{\PYZsq{}}\PY{l+s+s1}{Targeting Carrier (ISP)}\PY{l+s+s1}{\PYZsq{}}\PY{p}{,} \PY{l+s+s1}{\PYZsq{}}\PY{l+s+s1}{Targeting Geo \PYZhy{} Postal Code}\PY{l+s+s1}{\PYZsq{}}
          \PY{p}{]}
          
          \PY{c+c1}{\PYZsh{} get the column names that are not missing by design}
          \PY{n}{not\PYZus{}mbd} \PY{o}{=} \PY{n}{has\PYZus{}null}\PY{p}{[}\PY{n}{has\PYZus{}null}\PY{o}{.}\PY{n}{isin}\PY{p}{(}\PY{n}{mbd}\PY{p}{)} \PY{o}{==} \PY{k+kc}{False}\PY{p}{]}
          \PY{n}{not\PYZus{}mbd}
\end{Verbatim}


\begin{Verbatim}[commandchars=\\\{\}]
{\color{outcolor}Out[{\color{outcolor}135}]:} 0                        EndDate
          1     CandidateBallotInformation
          10                      Segments
          11                  LocationType
          17            CreativeProperties
          dtype: object
\end{Verbatim}
            
    \begin{Verbatim}[commandchars=\\\{\}]
{\color{incolor}In [{\color{incolor}136}]:} \PY{c+c1}{\PYZsh{} the proportion of nulls for the remaining columns}
          \PY{n}{prop\PYZus{}null}\PY{p}{[}\PY{n}{not\PYZus{}mbd}\PY{p}{]}
\end{Verbatim}


\begin{Verbatim}[commandchars=\\\{\}]
{\color{outcolor}Out[{\color{outcolor}136}]:} EndDate                       0.198607
          CandidateBallotInformation    0.931880
          Segments                      0.337269
          LocationType                  0.994550
          CreativeProperties            0.195883
          dtype: float64
\end{Verbatim}
            
    Based on the readme.txt file given with the data, it is fairly
reasonable to believe that the missingness in these columns can be
explained by: * \textbf{EndDate}: if null, then the advertisement is
ongoing * \textbf{CandidateBallotInformation}: if null, then the
advertisement was not for a political candidate * \textbf{Segments}: if
null, the advertiser did not provide a segment targeting criteria *
\textbf{LocationType}: if null, the advertiser did not specify a
location target * \textbf{CreativeProperties}: if null, the
advertisement did not link to a url

We will analyze the missingness of `EndDate'. Based on our assumption,
we would expect more recent advertisements to have more missing values.

    \begin{Verbatim}[commandchars=\\\{\}]
{\color{incolor}In [{\color{incolor}307}]:} \PY{c+c1}{\PYZsh{} create a table with relevant columns: start and end dates}
          \PY{n}{dates} \PY{o}{=} \PY{n}{ad\PYZus{}data}\PY{p}{[}\PY{p}{[}\PY{l+s+s1}{\PYZsq{}}\PY{l+s+s1}{StartDate}\PY{l+s+s1}{\PYZsq{}}\PY{p}{,} \PY{l+s+s1}{\PYZsq{}}\PY{l+s+s1}{EndDate}\PY{l+s+s1}{\PYZsq{}}\PY{p}{]}\PY{p}{]}\PY{o}{.}\PY{n}{copy}\PY{p}{(}\PY{p}{)}
          
          \PY{c+c1}{\PYZsh{} add a boolean column to dates that is True when EndDate is null and False otherwise}
          \PY{n}{dates}\PY{p}{[}\PY{l+s+s1}{\PYZsq{}}\PY{l+s+s1}{EndNull}\PY{l+s+s1}{\PYZsq{}}\PY{p}{]} \PY{o}{=} \PY{n}{dates}\PY{p}{[}\PY{l+s+s1}{\PYZsq{}}\PY{l+s+s1}{EndDate}\PY{l+s+s1}{\PYZsq{}}\PY{p}{]}\PY{o}{.}\PY{n}{isna}\PY{p}{(}\PY{p}{)}
          
          \PY{c+c1}{\PYZsh{} get the date of the earliest ad start}
          \PY{n}{earliest\PYZus{}ad} \PY{o}{=} \PY{n}{dates}\PY{p}{[}\PY{l+s+s1}{\PYZsq{}}\PY{l+s+s1}{StartDate}\PY{l+s+s1}{\PYZsq{}}\PY{p}{]}\PY{o}{.}\PY{n}{min}\PY{p}{(}\PY{p}{)}
          
          \PY{c+c1}{\PYZsh{} add time delta from the earliest start date in nanoseconds}
          \PY{c+c1}{\PYZsh{} this allows us to calculate statistics such as the mean}
          \PY{n}{dates}\PY{p}{[}\PY{l+s+s1}{\PYZsq{}}\PY{l+s+s1}{TimeDelta}\PY{l+s+s1}{\PYZsq{}}\PY{p}{]} \PY{o}{=} \PY{n}{dates}\PY{p}{[}\PY{l+s+s1}{\PYZsq{}}\PY{l+s+s1}{StartDate}\PY{l+s+s1}{\PYZsq{}}\PY{p}{]}\PY{o}{.}\PY{n}{apply}\PY{p}{(}\PY{k}{lambda} \PY{n}{start} \PY{p}{:} \PY{n}{pd}\PY{o}{.}\PY{n}{to\PYZus{}timedelta}\PY{p}{(}\PY{n}{start} \PY{o}{\PYZhy{}} \PY{n}{earliest\PYZus{}ad}\PY{p}{)}\PY{o}{.}\PY{n}{delta}\PY{p}{)}
          
          \PY{c+c1}{\PYZsh{} pd.to\PYZus{}timedelta(dates[\PYZsq{}EndDate\PYZsq{}].iloc[0] \PYZhy{} earliest\PYZus{}ad)}
          \PY{c+c1}{\PYZsh{} dates.groupby(\PYZsq{}EndNull\PYZsq{}).()}
          \PY{n}{mean\PYZus{}by\PYZus{}EndNull} \PY{o}{=} \PY{n}{dates}\PY{o}{.}\PY{n}{groupby}\PY{p}{(}\PY{l+s+s1}{\PYZsq{}}\PY{l+s+s1}{EndNull}\PY{l+s+s1}{\PYZsq{}}\PY{p}{)}\PY{o}{.}\PY{n}{mean}\PY{p}{(}\PY{p}{)}
          
          
          
          \PY{c+c1}{\PYZsh{} calculate observed statistic: difference in means}
          \PY{n}{obs} \PY{o}{=} \PY{n}{mean\PYZus{}by\PYZus{}EndNull}\PY{o}{.}\PY{n}{diff}\PY{p}{(}\PY{p}{)}\PY{o}{.}\PY{n}{iloc}\PY{p}{[}\PY{o}{\PYZhy{}}\PY{l+m+mi}{1}\PY{p}{]}\PY{o}{.}\PY{n}{values}
          
          \PY{c+c1}{\PYZsh{} display the average time delta grouped by EndNull}
          \PY{n}{mean\PYZus{}by\PYZus{}EndNull}
\end{Verbatim}


\begin{Verbatim}[commandchars=\\\{\}]
{\color{outcolor}Out[{\color{outcolor}307}]:}                  TimeDelta
          EndNull                   
          False    28335144985266584
          True     33283332955792772
\end{Verbatim}
            
    It is difficult to assess the difference in nanoseconds, so let us find
the difference as a timedelta.

    \begin{Verbatim}[commandchars=\\\{\}]
{\color{incolor}In [{\color{incolor}212}]:} \PY{c+c1}{\PYZsh{} take the difference in nanoseconds and convert to timedelta}
          \PY{n}{pd}\PY{o}{.}\PY{n}{to\PYZus{}timedelta}\PY{p}{(}\PY{n}{mean\PYZus{}by\PYZus{}EndNull}\PY{o}{.}\PY{n}{diff}\PY{p}{(}\PY{p}{)}\PY{o}{.}\PY{n}{iloc}\PY{p}{[}\PY{o}{\PYZhy{}}\PY{l+m+mi}{1}\PY{p}{]}\PY{p}{)}
\end{Verbatim}


\begin{Verbatim}[commandchars=\\\{\}]
{\color{outcolor}Out[{\color{outcolor}212}]:} TimeDelta   57 days 06:29:47.970526
          Name: True, dtype: timedelta64[ns]
\end{Verbatim}
            
    We will now perform a permutation test to determine if `EndDate' is
dependent on `StartDate'.

We set a significance level of 0.01

    \begin{Verbatim}[commandchars=\\\{\}]
{\color{incolor}In [{\color{incolor}334}]:} \PY{c+c1}{\PYZsh{} let N be the number of repetitions for the permutation test}
          \PY{n}{N} \PY{o}{=} \PY{l+m+mi}{500}
          
          \PY{c+c1}{\PYZsh{} list to store the simulated statistics}
          \PY{n}{stats} \PY{o}{=} \PY{p}{[}\PY{p}{]}
          \PY{k}{for} \PY{n}{\PYZus{}} \PY{o+ow}{in} \PY{n+nb}{range}\PY{p}{(}\PY{n}{N}\PY{p}{)}\PY{p}{:}
              
              \PY{c+c1}{\PYZsh{} shuffle the EndNull column}
              \PY{n}{shuffled\PYZus{}EndNull} \PY{o}{=} \PY{p}{(}
                  \PY{n}{dates}\PY{p}{[}\PY{l+s+s1}{\PYZsq{}}\PY{l+s+s1}{EndNull}\PY{l+s+s1}{\PYZsq{}}\PY{p}{]}
                  \PY{o}{.}\PY{n}{sample}\PY{p}{(}\PY{n}{replace}\PY{o}{=}\PY{k+kc}{False}\PY{p}{,} \PY{n}{frac}\PY{o}{=}\PY{l+m+mi}{1}\PY{p}{)}
                  \PY{o}{.}\PY{n}{reset\PYZus{}index}\PY{p}{(}\PY{n}{drop}\PY{o}{=}\PY{k+kc}{True}\PY{p}{)}
              \PY{p}{)}
              
              \PY{c+c1}{\PYZsh{} put them in a table}
              \PY{n}{shuffled} \PY{o}{=} \PY{p}{(}
                  \PY{n}{dates}\PY{o}{.}\PY{n}{assign}\PY{p}{(}\PY{o}{*}\PY{o}{*}\PY{p}{\PYZob{}}
                      \PY{l+s+s1}{\PYZsq{}}\PY{l+s+s1}{EndNull}\PY{l+s+s1}{\PYZsq{}}\PY{p}{:} \PY{n}{shuffled\PYZus{}EndNull}\PY{p}{,}
                      \PY{l+s+s1}{\PYZsq{}}\PY{l+s+s1}{TimeDelta}\PY{l+s+s1}{\PYZsq{}}\PY{p}{:} \PY{n}{dates}\PY{p}{[}\PY{l+s+s1}{\PYZsq{}}\PY{l+s+s1}{TimeDelta}\PY{l+s+s1}{\PYZsq{}}\PY{p}{]}
                  \PY{p}{\PYZcb{}}\PY{p}{)}
              \PY{p}{)}
              
              \PY{c+c1}{\PYZsh{} compute the differences in means}
              \PY{c+c1}{\PYZsh{} gets the time delta in nanoseconds as a float}
              \PY{n}{mean} \PY{o}{=} \PY{n}{shuffled}\PY{o}{.}\PY{n}{groupby}\PY{p}{(}\PY{l+s+s1}{\PYZsq{}}\PY{l+s+s1}{EndNull}\PY{l+s+s1}{\PYZsq{}}\PY{p}{)}\PY{o}{.}\PY{n}{mean}\PY{p}{(}\PY{p}{)}\PY{o}{.}\PY{n}{diff}\PY{p}{(}\PY{p}{)}\PY{o}{.}\PY{n}{iloc}\PY{p}{[}\PY{o}{\PYZhy{}}\PY{l+m+mi}{1}\PY{p}{]}\PY{o}{.}\PY{n}{values}\PY{p}{[}\PY{l+m+mi}{0}\PY{p}{]}
              \PY{n}{stats}\PY{o}{.}\PY{n}{append}\PY{p}{(}\PY{n}{mean}\PY{p}{)}
\end{Verbatim}


    \begin{Verbatim}[commandchars=\\\{\}]
{\color{incolor}In [{\color{incolor}335}]:} \PY{c+c1}{\PYZsh{} calculate the p\PYZus{}value: how many differences are greater than }
          \PY{n}{p\PYZus{}value} \PY{o}{=} \PY{p}{(}\PY{n}{stats} \PY{o}{\PYZgt{}} \PY{n}{obs}\PY{p}{)}\PY{o}{.}\PY{n}{mean}\PY{p}{(}\PY{p}{)}
          \PY{n}{p\PYZus{}value}
\end{Verbatim}


\begin{Verbatim}[commandchars=\\\{\}]
{\color{outcolor}Out[{\color{outcolor}335}]:} 0.0
\end{Verbatim}
            
    With a p-value of 0.0 and a significance level of 0.01, we can assume
that the missingness of EndDate is dependent on StartDate.

    \hypertarget{hypothesis-test}{%
\subsubsection{Hypothesis Test}\label{hypothesis-test}}

    Earlier, we noted how specifying a target age had roughly no effect on
relationship between `Spend' and `Impressions'. However, the plot shows
that ads specifying a target age have slightly more impressions per
spending than ads that do not specify a target age. We will perform a
hypothesis test to rigorously determine if specifying a target age does
improve the number of impressions per spending.

\hypertarget{null-hypothesis}{%
\paragraph{Null Hypothesis}\label{null-hypothesis}}

Specifying a target age has roughly no effect on the relationship
between `Spend' and `Impressions'.

\hypertarget{alternative-hypothesis}{%
\paragraph{Alternative Hypothesis:}\label{alternative-hypothesis}}

Specifying a target age increases the number of impressions per
spending.

\hypertarget{test-statistic}{%
\paragraph{Test Statistic:}\label{test-statistic}}

The test statistic we will be using is the mean number of impressions
per dollar (USD) spent for ads that specified a target age.

\textbf{Significance Level:} 0.05

    \begin{Verbatim}[commandchars=\\\{\}]
{\color{incolor}In [{\color{incolor}389}]:} \PY{c+c1}{\PYZsh{} we will use the data without outliers, as that was the data used in the plot}
          
          \PY{c+c1}{\PYZsh{} select the relevant columns}
          \PY{n}{age\PYZus{}data} \PY{o}{=} \PY{n}{data\PYZus{}no\PYZus{}out}\PY{p}{[}\PY{p}{[}\PY{l+s+s1}{\PYZsq{}}\PY{l+s+s1}{Spend}\PY{l+s+s1}{\PYZsq{}}\PY{p}{,} \PY{l+s+s1}{\PYZsq{}}\PY{l+s+s1}{Impressions}\PY{l+s+s1}{\PYZsq{}}\PY{p}{,} \PY{l+s+s1}{\PYZsq{}}\PY{l+s+s1}{AgeBracket}\PY{l+s+s1}{\PYZsq{}}\PY{p}{]}\PY{p}{]}\PY{o}{.}\PY{n}{copy}\PY{p}{(}\PY{p}{)}
          
          \PY{c+c1}{\PYZsh{} add a column \PYZsq{}NullAge\PYZsq{} that is True if the ad has a null value in \PYZsq{}AgeBracket\PYZsq{} and False otherwise}
          \PY{n}{age\PYZus{}data}\PY{p}{[}\PY{l+s+s1}{\PYZsq{}}\PY{l+s+s1}{NullAge}\PY{l+s+s1}{\PYZsq{}}\PY{p}{]} \PY{o}{=} \PY{n}{spend\PYZus{}imp\PYZus{}by\PYZus{}age}\PY{p}{[}\PY{l+s+s1}{\PYZsq{}}\PY{l+s+s1}{AgeBracket}\PY{l+s+s1}{\PYZsq{}}\PY{p}{]}\PY{o}{.}\PY{n}{isna}\PY{p}{(}\PY{p}{)}
          
          \PY{c+c1}{\PYZsh{} since some ads spent 0 dollars, drop those values (there are 60 of those ads)}
          \PY{n}{age\PYZus{}data} \PY{o}{=} \PY{n}{age\PYZus{}data}\PY{p}{[}\PY{n}{age\PYZus{}data}\PY{p}{[}\PY{l+s+s1}{\PYZsq{}}\PY{l+s+s1}{Spend}\PY{l+s+s1}{\PYZsq{}}\PY{p}{]} \PY{o}{\PYZgt{}} \PY{l+m+mi}{0}\PY{p}{]}
          
          \PY{c+c1}{\PYZsh{} add a column containing the number of impressions per dollar spent for each ad}
          \PY{n}{age\PYZus{}data}\PY{p}{[}\PY{l+s+s1}{\PYZsq{}}\PY{l+s+s1}{ImpPerSpend}\PY{l+s+s1}{\PYZsq{}}\PY{p}{]} \PY{o}{=} \PY{n}{age\PYZus{}data}\PY{p}{[}\PY{l+s+s1}{\PYZsq{}}\PY{l+s+s1}{Impressions}\PY{l+s+s1}{\PYZsq{}}\PY{p}{]} \PY{o}{/} \PY{n}{age\PYZus{}data}\PY{p}{[}\PY{l+s+s1}{\PYZsq{}}\PY{l+s+s1}{Spend}\PY{l+s+s1}{\PYZsq{}}\PY{p}{]}
          
          
          \PY{c+c1}{\PYZsh{} display the table}
          \PY{n}{age\PYZus{}data}\PY{o}{.}\PY{n}{head}\PY{p}{(}\PY{p}{)}
\end{Verbatim}


\begin{Verbatim}[commandchars=\\\{\}]
{\color{outcolor}Out[{\color{outcolor}389}]:}    Spend  Impressions AgeBracket  NullAge  ImpPerSpend
          0   1044       137185       35++    False   131.403257
          1    279        94161        18+    False   337.494624
          2   6743      3149886        NaN     True   467.134213
          3   3698       573475        18+    False   155.077069
          4    445       232906        25+    False   523.384270
\end{Verbatim}
            
    \begin{Verbatim}[commandchars=\\\{\}]
{\color{incolor}In [{\color{incolor}393}]:} \PY{c+c1}{\PYZsh{} calculate the observed statistic}
          
          \PY{n}{grouped\PYZus{}age} \PY{o}{=} \PY{n}{age\PYZus{}data}\PY{o}{.}\PY{n}{groupby}\PY{p}{(}\PY{l+s+s1}{\PYZsq{}}\PY{l+s+s1}{NullAge}\PY{l+s+s1}{\PYZsq{}}\PY{p}{)}\PY{o}{.}\PY{n}{mean}\PY{p}{(}\PY{p}{)}
          
          \PY{c+c1}{\PYZsh{} calculate the observed statistic}
          \PY{c+c1}{\PYZsh{} if NullAge is False, the ad specified a target age}
          \PY{n}{obs} \PY{o}{=} \PY{n}{grouped\PYZus{}age}\PY{o}{.}\PY{n}{loc}\PY{p}{[}\PY{k+kc}{False}\PY{p}{,} \PY{l+s+s1}{\PYZsq{}}\PY{l+s+s1}{ImpPerSpend}\PY{l+s+s1}{\PYZsq{}}\PY{p}{]}
          
          \PY{c+c1}{\PYZsh{} display the observed statistic}
          \PY{n}{obs}
\end{Verbatim}


\begin{Verbatim}[commandchars=\\\{\}]
{\color{outcolor}Out[{\color{outcolor}393}]:} 453.5088083470662
\end{Verbatim}
            
    \begin{Verbatim}[commandchars=\\\{\}]
{\color{incolor}In [{\color{incolor}404}]:} \PY{c+c1}{\PYZsh{} let N be the number of simulations}
          \PY{n}{N} \PY{o}{=} \PY{l+m+mi}{10000}
          
          \PY{c+c1}{\PYZsh{} store each simulation\PYZsq{}s result in a list}
          \PY{n}{results} \PY{o}{=} \PY{p}{[}\PY{p}{]}
          
          \PY{c+c1}{\PYZsh{} get the number of ads that specified a target age}
          \PY{c+c1}{\PYZsh{} the number of ads minus the number of ads that did not specify a target age}
          \PY{n}{num\PYZus{}specified} \PY{o}{=} \PY{n}{age\PYZus{}data}\PY{p}{[}\PY{l+s+s1}{\PYZsq{}}\PY{l+s+s1}{NullAge}\PY{l+s+s1}{\PYZsq{}}\PY{p}{]}\PY{o}{.}\PY{n}{size} \PY{o}{\PYZhy{}} \PY{n}{age\PYZus{}data}\PY{p}{[}\PY{l+s+s1}{\PYZsq{}}\PY{l+s+s1}{NullAge}\PY{l+s+s1}{\PYZsq{}}\PY{p}{]}\PY{o}{.}\PY{n}{sum}\PY{p}{(}\PY{p}{)}
          
          \PY{k}{for} \PY{n}{\PYZus{}} \PY{o+ow}{in} \PY{n+nb}{range}\PY{p}{(}\PY{n}{N}\PY{p}{)}\PY{p}{:}
              \PY{c+c1}{\PYZsh{} take a sample the size of num\PYZus{}specified of ads\PYZsq{} ImpPerSpend}
              \PY{n}{simulation} \PY{o}{=} \PY{n}{np}\PY{o}{.}\PY{n}{random}\PY{o}{.}\PY{n}{choice}\PY{p}{(}\PY{n}{age\PYZus{}data}\PY{p}{[}\PY{l+s+s1}{\PYZsq{}}\PY{l+s+s1}{ImpPerSpend}\PY{l+s+s1}{\PYZsq{}}\PY{p}{]}\PY{p}{,} \PY{n}{size}\PY{o}{=}\PY{n}{num\PYZus{}specified}\PY{p}{,} \PY{n}{replace}\PY{o}{=}\PY{k+kc}{True}\PY{p}{)}
              \PY{c+c1}{\PYZsh{} add the mean to results}
              \PY{n}{results}\PY{o}{.}\PY{n}{append}\PY{p}{(}\PY{n}{np}\PY{o}{.}\PY{n}{mean}\PY{p}{(}\PY{n}{simulation}\PY{p}{)}\PY{p}{)}
\end{Verbatim}


    \begin{Verbatim}[commandchars=\\\{\}]
{\color{incolor}In [{\color{incolor}405}]:} \PY{c+c1}{\PYZsh{} calculate the p\PYZus{}value}
          \PY{n}{p\PYZus{}value} \PY{o}{=} \PY{p}{(}\PY{n}{results} \PY{o}{\PYZgt{}} \PY{n}{obs}\PY{p}{)}\PY{o}{.}\PY{n}{mean}\PY{p}{(}\PY{p}{)}
          \PY{n}{p\PYZus{}value}
\end{Verbatim}


\begin{Verbatim}[commandchars=\\\{\}]
{\color{outcolor}Out[{\color{outcolor}405}]:} 0.9915
\end{Verbatim}
            
    With a p-value of 0.9915, we fail to reject the null hypothesis. In
fact, it seems more likely that not specifying a target age-group
increases the number of impressions per dollar spent. There is likely a
confounding factor for this result: ads that did not specify a target
age also spent far less on the advertisement.

The issue may lie with the test statistic; perhaps the number of
impressions per dollar spent strongly favors ads that spent less money.

    \hypertarget{further-exploration}{%
\section{Further Exploration}\label{further-exploration}}

Our hypothesis test had a rather surprising result. Because of this, we
should analyze the ads that paid little to no money. How were they able
to advertise? It is likely that they share some special characteristic,
such as being a charity or non-profit ad.

Here, we found that spending had a positive correlation with
impressions. We could identify the data points with high and low
residuals to search for defining characteristics. To be very thorough,
this would involve viewing the ad itself and quantifying visual and
aural data. Other characteristics to explore could be the duration of
the ad, the color scheme, the inclusion of music, the presence of a
human face, or the use of on-screen text.


    % Add a bibliography block to the postdoc
    
    
    
    \end{document}
